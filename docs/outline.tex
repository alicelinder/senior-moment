\documentclass[11pt]{article}
\usepackage{textcomp}
\usepackage{fontenc}
\usepackage{ amssymb }
\usepackage{amsmath} % for gathering multiple equations in one group
\setlength{\parskip}{3 mm}

\renewcommand*{\familydefault}{\sfdefault}

\usepackage{outlines} % for Roman/alpha style outline
\usepackage{enumitem}
\setenumerate[1]{label=\Roman*.}
\setenumerate[2]{label=\Alph*.}
\setenumerate[3]{label=\roman*.}
\setenumerate[4]{label=\alph*.}

\begin{document}
\begin{outline}[enumerate]
\1 Introduction
	\2 Range extremes and community ecology
		\3 Ackerly 2003: Community assembly as an ecological sorting process
			\4 plants in saturated communities occupy optimal or near-optimal environments
			\4 distribution is due to abiotic and biotic factors
			\4 plasticity within a species alters functional traits
		\3 Climate change's influence on range extremes
			\4 Ackerly 2003: population's responses to changing environments
				\item Shifts in the optimal environment: microhabitat and/or altitude
				\item Large-scale shifts in geographic distribution (range expansion and extirpation at "trailing edge")
			\4 Chen et. al 2011
	\2 Plants at their edge at a disadvantage $\rightarrow$ how does this become apparent in their functional traits and competitiveness?
		\3 Ackerly 2003: Trailing edge hypothesis ? climate change's rate affects how prevalent adaptive responses are
			\4 Historically, species replacement (migration) outpace local adaptation
			\4 Leading edge will become either site of extinction or site of migration
			\4 Plants at range extremes outside optimal environment
				\item can't track preferred environment simultaneously in multiple dimensions, thus effective dispersal will cause species to track preferred conditions
	\2 Altered opportunities for trees at their range extremes
		\3 Look at range extremes to predict future responses to climate change
	\2 Hypotheses
		\3 H1: Less competitive as you approach (climatic and/or latitudinal) range limit (plant at disadvantage[EW1])
			\4 A1: More competitive as you approach (climatic and/or latitudinal) range limit[EW2]
			\4 A2: Competition not predicted by position in (climatic and/or latitudinal) range (and if not, could they differ across and within the range, and could there be an explanation for this?)[EW3]
		\3 H2: Trees at their range extremes exhibit altered suites of functional traits from trees in the range interior
			\4 A1: Trait variation increases as plants approach their range extremes (indicates plants trying many different combinations of traits)
			\4 A2: Trait variation decreases as plants approach range extremes
			\4 A3: Trait means change across range

\1 Methods (draft)
	\2 At three different field sites across the latitudinal gradient, I examined 6 different species of woody plants with 6 different individuals from each species. The species in question are: Acer pensylvanicum, Betula papyrifera, Cornus alternifolia, Fagus grandifolia, Hamamelis virginiana, and Sorbus americana. All species are present in some capacity in at least one of the three sites. This was determined by the Wolkovich lab, which has tagged a number of individuals of these species, collected seeds from many, mapped them, and monitored them since about Fall 2015. The species in question have either a northern or southern range limit close to one of the three sites, determined through the websites bonap.org, plants.usda.gov, and gbif.org. The three sites were located at Harvard Forest, near the Passaconaway Campground in the White Mountains of New Hampshire, and near Saint Hippolyte in Quebec. Using an individual of each species as the center of a circular plot, I measured a circle with a radius of 5m using a 100ft meter tape, with that individual of interest as the center of the circle. I measured the DBH of any trees or plants within that circle (where applicable), and took note of the presence/absence of any other species in the circular plot. For functional traits, I used the data collected by Harry Stone during the Harvard Forest Summer Program 2015. The height of individuals not measured summer 2015 was measured through a clinometer. Additionally, I measured the DBH of any newly-tagged individuals (in cases where an individual is too close to another focal plant). In the field, I took notes on datasheets, noting the presence and absence of different species and the DBH of nearby individuals (when applicable). Functional traits for newly-tagged individuals were not measured, but instead the traits measured in 2015 will served as a representation of the traits of all the species within that site.

\1 Results (expected figures)
	\2 Fraction Basal Area vs. Functional Trait (need to figure out some way to graph this?how do I organize Functional Traits to indicate that one is better or worse for a plant? Is that possible?)
	\2 Fraction Basal Area vs. Position in Range (Climatic and latitudinal)
	\2 Possibly map of climatic ranges?
	\2 Trait means vs. Fraction Basal Area
	\2 All-Species dominance in plot vs. other species in plot (I remember discussing this over the summer?is there any reason to tie this in?)
	\2 Is there any reason to include MDS of species composition? Might be interesting to see how my 4 sites varied, though not sure if it's super relevant to the direction of my paper?
	\2 Coefficient of variation for traits vs. fraction basal area

\end{outline}

Working Bibliography:

Ackerly, David�D. "Community Assembly, Niche Conservatism, and Adaptive Evolution in Changing Environments." International journal of plant sciences 164 (2003): S165-84. Print.

Chen, I-C, et al. "Rapid Range Shifts of Species Associated with High Levels of Climate Warming.(REPORTS)(Author Abstract)." Science 333.6045 (2011): 1024. Print.

\end{document}